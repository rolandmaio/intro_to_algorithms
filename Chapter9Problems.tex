\documentclass[10pt,a4paper]{article}
\usepackage[latin1]{inputenc}
\usepackage{amsmath}
\usepackage{mathtools}
\usepackage{amsfonts}
\usepackage{amssymb}
\usepackage{algorithm}
\usepackage[noend]{algpseudocode}

\title{Problems Chapter 9}
\author{Roland Maio}
\date{\today}

\begin{document}
\maketitle
\section*{Exercise 9.2-1}
Show that RANDOMIZED-SELECT never makes a recursive call to a 0-length array.
\vspace*{\baselineskip}
\\
Assume, to the contrary, that RANDOMIZED-SELECT does make a recursive call to a 0-length array. Then, this recursive call must occur on line 8 or 9. If it occurs on line 8, then it must be the case that $q=p$ after the call to RANDOMIZED-PARTITION on line 3. In which case, $k=1$ and $i\neq 1$, since otherwise we would have returned $A[q]$ on line 6. But since, $k=1\neq i$, it cannot be the case that $i<k$ and so the recursive call to RANDOMIZED-SELEECT on line 8 is not executed. But this contradicts our assumption. Otherwise, the recursive call to RANDOMIZED-SELECT to a 0-length array must occur on line 9. In which case we must have $q=r$. By the similar reasoning as above, however, if $q=r$, then we must have $i<k$ and so the recursive call on line 9 cannot possibly execute. Therefore, RANDOMIZED-SELECT does not make a recursive call to a 0-length array.

\section*{Exercise 9.2-2}
Argue that the indicator random variable $X_k$ and the value $T(\textit{max}(k-1,n-k))$ are independent.
\vspace*{\baselineskip}
\\
Consider the the expression $\textit{max}(k-1, n-k)$ which is defined as
\begin{equation}
\textit{max}(k-1,n-k)=\begin{cases}
	k-1 & \text{if } k > \lceil n/2\rceil\\
	n-k & \text{if } k \leq \lceil n/2\rceil
	\end{cases}
\end{equation}
and let $Y$ be the random variable that is the value of $\textit{max}(k-1,n-k)$. Observe, that $P\lbrace Y=k-1\rbrace = P\lbrace k > \lceil n/2 \rceil\rbrace = \frac{n-\lceil n/2\rceil}{n}$. This probability is independent of the value of $k$. And since $P\lbrace Y=n-k\rbrace = 1 - P\lbrace Y=k-1\rbrace$ we conclude that the random variable $Y$ is independent of the value of $k$ for all $k$. And since $X_k$ is entirely dependent on this quantity, it follows that $Y$ is independent of $X_k$ and finally that the random variables $X_k$ and the value $T(\textit{max}(k-1,n-k))$ are independent.

\section*{Exercise 9.2-3}
Write an iterative version of RANDOMIZED-SELECT.
\vspace*{\baselineskip}
\\
Cows...

\end{document}